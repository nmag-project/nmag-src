%======================================================================
% P R E A M B L E
%======================================================================
\documentclass[12pt,twoside]{article}
%----------------------------------------------------------------------
% Reprocess only those files which have changed recently:
% \includeonly{intro,creating,commands}
%----------------------------------------------------------------------
% Create a listing in the log of all files needed to process this document
\listfiles
%----------------------------------------------------------------------
% Non-standard packages I am using (things I ve written, borrowed, etc.):
%\usepackage{} % ... note that old .sty files can be included here
% Standard LaTeX2e packages I am using (as seen in "The LaTeX Companion"):
\usepackage[dvips]{graphicx}
% ... if you want to include encapsulated postscript figures
\usepackage{makeidx}
% ... if you want an index
\usepackage{amsmath}
\usepackage{amssymb}
\usepackage{subfigure}
% ... if you need lots of math symbols
\usepackage[dvips=true,bookmarks=true]{hyperref}
% ... only needed for PDF generation
\usepackage{epsfig}
%----------------------------------------------------------------------
\makeindex
% activate index-making
%----------------------------------------------------------------------
% The following command sets "1.2" as the line spacing throughout the
% thesis for readability (optional).
\renewcommand{\baselinestretch}{1.2}
%----------------------------------------------------------------------
% Reset page margins properly for doublesided pages
\setlength{\marginparwidth}{0pt}
\setlength{\marginparsep}{0pt}
\setlength{\oddsidemargin}{0.125in}
\setlength{\evensidemargin}{0.125in}
\setlength{\textwidth}{6.375in}
\raggedbottom
%----------------------------------------------------------------------
% My own command and environment definitions:
\newcommand{\om}[1]{(\texttt{#1})}
%\newcommand{\om}[1]{}

\renewcommand{\topfraction}{0.85}
\renewcommand{\textfraction}{0.1}
\renewcommand{\floatpagefraction}{0.75}

\newcommand{\program}[1]{\textbf{#1}} % program names in bold text
\newcommand{\exten}[1]{\texttt{#1}} % file extensions in bold text
% (use caps)
\newcommand{\cmmd}[1]{\textbackslash\texttt{#1}} % command name in tt font
\renewcommand{\d}{\textrm{d}} % integration varable

\newcommand{\beq}{\begin{equation}} % begin equation
\newcommand{\eeq}{\end{equation}} % end equation
\newcommand{\enviro}[1]{\texttt{#1}} % environment name in tt font
\newcommand{\eg}{\textit{e.g.},} % some Latin abreviations in italic
\newcommand{\ie}{\textit{i.e.},} \newcommand{\etc}{\textit{etc}.\@}

\newcommand{\mat}[1]{\ensuremath{\mathcal{#1}}} % matrix names in
% uppercase caligraphic
\newcommand{\vect}[1]{\ensuremath{\mathit{#1}}} % vector names in math
% italic
\newcommand{\rv}[1]{\ensuremath{\mathbf{#1}}} % math bold for random
% variables 
\newcommand{\degg}[1]{\mbox{\raisebox{3pt}{$\circ$}\hspace{-.5pt}#1}}
% command to produce a degree sign. Example: \degg[C] gives degrees Celcius

\newenvironment{definition}[1]{\begin{quote}\emph{#1}:}{\end{quote}}
% Provides indented formal definition and emphasizes the word.
% e.g. \begin{definition}{Reliability} ... \end{definition}

\newenvironment{where}[1]% Equation symbol lists
{\begin{list}{}%
{\renewcommand{\makelabel}[1]{\hfill\textnormal{##1 =}}%
\settowidth{\labelwidth}{\textnormal{#1 =}}%
\setlength{\leftmargin}{\labelwidth}%
\addtolength{\leftmargin}{\labelsep}%
\setlength{\itemsep}{-\parsep}}}%
{\end{list}}

%\nofiles

% Example:
% \begin{where}{where $E$}
% \item[where $E$] least squares error term;
% \item[$w$] weighting factor associated with each measured variable.
% \end{where}

%======================================================================
% L O G I C A L D O C U M E N T
%======================================================================
\begin{document}
%----------------------------------------------------------------------
% FRONT MATERIAL
%----------------------------------------------------------------------
\pagestyle{empty}
%\include{titlepage}
%\include{titlepage}
% Title page, declaration, borrowers  page, abstract, acknowlegements,
% dedication, table of contents, list of tables, list of figures

%----------------------------------------------------------------------
% MAIN BODY
%----------------------------------------------------------------------
% Put the document title and page numbers in the header
%\pagestyle{myheadings}
%\tableofcontents
\pagestyle{plain}
%\listoffigures
%\listoftables
%\newpage
% Put title on left & chapter heading goes on right by default
%\markboth{\LaTeX\ for Thesis and Large Documents}
% Go to normal sized type
\normalsize
% Chapters
% Include your "sub" source files here (must have extension .tex)
% change the page numbering style to arabic
\pagenumbering{arabic}
%\include{main}
%\include{introduction} %
%\include{theory_micromagnetism} %
%\include{numerical_methods} %
%\include{square_rings} %
%\include{2d_antidots} %
%\include{mesh_generation} %
%\include{AMR} %
%\include{summary} %
%----------------------------------------------------------------------
% END MATERIAL
%----------------------------------------------------------------------
%% Appendices
%\appendix
%% Designate with \appendix declaration which just changes numbering style
%% from here on

%% Glossary
%% You could use a \begin{description} ... \end{description} for this
%\include{glossary}

%% Bibliography
%% If done using the BibTeX program, use
%\bibliographystyle{unsrt} % sorted alphabetically, labeled with numbers
%\bibliography{reportbib} % names file keylatex.bib as my bibliography file
%% OR, do it "by hand" inside a "thebibliography" enivironment

% Index % Put a \makeindex command in the Preamble if you use
% MakeIndex program
% and put
%\printindex % here
% OR, do it "by hand" inside \begin{theindex} ... \end{theindex}
%----------------------------------------------------------------------
%======================================================================

The mesher algorithm is composed of two parts. In the first one
\om{pyfem.ml}, directly coupled to the python interface, the various
objects that can be defined in python (ellipsoids, frustums, boxes)
are translated in a form suitable to be handled by the mesher
itself \om{bc\_ellipsoid, bc\_frustum, bc\_box}. An object is
represented by a function that has value $> 0$ in its interior
and $< 0$ in the exterior. Any geometrical transformation
(rotation, scaling, shift) on the object is expressed as an affine
transformation on the coordinates defining the kernel of the function
and the union, intersection and difference of objects are defined with the
operators $\min$, $\max$ and function inversion. Together with the
bounding box, these functions are stored in the fem\_geometry
data structure. A further function\om{,the callback one} , which will
be passed to the driver and executed every n steps, is defined at this
initial stage, and its arguments are the piece number, the iteration
number and the mesh structure. The piece number is the position of the
object in the list of objects to mesh and the mesh structure has been
defined in the nmesh manual. The driver and the fem\_geometry are
passed to the function \om{mesh\_boundaries\_and\_objects} which
initialize the mesh generation. The default values of the parameters used in the
generation are stored in the mdefaults data structure, and optional
mobile points, fixed points or periodicity along some direction are
passed as well to the function \om{mesh\_boundaries\_and\_objects}.
%########################### add cache name ######################
%########################### add density ######################
Depending on the status of the periodic boundary flags, this function
performs two operations. If the periodic boundaries are selected, it
calls the function \om{mesh\_periodic\_outer\_box} which, for every
`periodic' axis of the space, creates a one dimensional mesh and
copies the resulting points to all the edges of the boundary box
along that axis. These points are then set as immobile and passed to the
next function, \om{mesh\_it\_work,} which would be directly called if
periodic boundaries are not specified. This function iterates over all
the objects and for each creates a mesh. The procedure is as follows:
for each of the pieces the function \om{add\_meshdata\_of\_piece\_nr} is
called with the number of piece, the list of simplices and coordinates
so far and the list of immobile points as arguments. This function
returns new lists of simplices and coordinates as well as a new list
of immobile points. When also the last piece is meshed, the function
\om{mesh\_from\_pieces\_meshdata} is called. The function
\om{add\_meshdata\_of\_piece\_nr} contains the function \om{mesh\_a\_piece},
which is called passing a random generator, the driver, mobile
and immobile points so far, the data from the fem\_geometry data
structure, the mesher default parameters and the chosen value of the
rod length. This functions return a status which can be
Mesh\_Engine\_Finished\_Step\_Limit\_Reached or
Mesh\_Engine\_Finished\_Force\_Equilibrium\_Reached, and a mesh. The
status identifies the reason way the mesher stopped, while the partial
mesh is `completed' \om{through the mesh\_grow\_bookkeping\_data
  function,} with informations about which simplices are within its
boundary and which nodes are on the boundary, seen as immobile
from this point on. When the iteration on these
\om{add\_meshdata\_of\_piece\_nr} calls is finished, the call to
\om{mesh\_from\_pieces\_meshdata} creates the final mesh through
the function \om{mesh\_from\_known\_delaunay} and returns it to the
function \om{ \_py\_mesh\_bodies\_raw} which is called from python.
Going back to the function \om{mesh\_a\_piece} called for each of the
objects to mesh, its design is the following. The geometry of the
object passed with \om{fem\_geometry} data structure is extracted as
a boundary conditions function and a corrected density function (
defined as 0 outside the object). Depending on the choice to use a
back-reference to an object already defined that we want to reproduce
(possibly transformed) or a brand new object, the function
\om{make\_meshgen\_engine} is called with different arguments. In the
first case the same or transformed points present in its reference are
used as starting points for the mesh relaxation, in the second case a
set of random points is generated for the same purpose. If in the
python script some mobile points are passed as argument to the
function \om{mesh\_bodies\_raw}, these are used instead of the random
generated ones. The call to the function \om{make\_meshgen\_engine}
starts a new set of operations which is the core of the mesher, and
returns an engine status which is interpreted by the driver as
previously explained.  The operations defined on a mesh are the
following: extract\_mesh, retriangulate, mirror\_simplices, change\_points,
determine\_forces, time\_step and do\_command. The handling of these
functions is performed by the engine. The extract\_mesh function is
the last operation and what it does is to retriangulate the final set
of points and return the mesh with all the information about its
geometry. The retriangulate function uses the current set of points to
obtain a series of simplices which are divided in relevant and
irrelevant \om{through the simplex\_classificator}: the first ones are
subject to neighbour and shape forces, the second ones to ad-hoc
forces that try to correct possible problems in the relevant
simplices. The mirror\_simplices function is designed 
to work only in the 2D and 3D cases so far. In 2D the algorithm, given a
simplex, separates the points on the boundaries from those immobiles
and in the interior of the object \om{filter\_points} and if the number
of boundary points is equal to (dim-1) and the internal point is
unique, this one is mirrored with respect to the plane defined by the
two boundary points \om{node\_action\_2D}. If an immobile point is
present in the simplex, such mirroring doesn't occur. In the 3D case,
the algorithm separates the points as well \om{filter\_points} and for
all the couples of boundary points a new point is inserted at half-way
between those \om{node\_action\_3D\_all}. In both 2D and 3D the surfaces
interested by a mirroring are progressively stored in a hashtable
\om{mirror\_on\_surfaces\_htb} so that possible problems related to the
density of points is avoided. The change\_points function takes an
array of status associated to each node and depending on that 
a node is left unchanged \om{Do\_Nothing\_With\_Point}, deleted
\om{Delete\_Point} or a new point is inserted in a random position
\om{add\_random\_point\_close\_to} around the node itself. The insertion
of points is performed such that the status \om{Boundary, Immobile,
  Mobile} of each of the already present nodes is unchanged. 
The function determine\_forces is the one where most of the numerical
computation is performed. For each simplex among the relevant ones a
neighbour force expands the points in order to fill all the space,
acting along the connections between the points, a shape force tries
to keep the simplex as regular as possible, and a volume force tries
to keep the volume as ideal as possible. All these forces are
described in the nmesh manual. In the determine\_forces function the
simplices with a volume/ideal volume ratio smaller than a \om{0.1}
threshold and where the distance between one of its points and the
plane defined by all the others is smaller than a \om{0.2} second
threshold are defined as slivers. Inside this function the irrelevant
simplices are used to correct the imperfections of the relevant
simplices. This is done using a force that acts on the nodes of the
irrelevant simplices and push them towards the centre of mass of the
irrelevant simplex. A further force tries to reduce the volume of the
slivers pushing their nodes towards the centre of mass of the sliver. 
The neighbour force is used as one of the conditions to insert or
delete a node of the mesh, that is if the sum of moduli of the
neighbour forces on the node is smaller than a \om{0.02} threshold a
node is inserted with probability 20\%. Another parameter used to
insert or delete a point is the voronoi volume associated to each
node, which is defined in the nmesh manual and whose value is passed
to the controller. At this point the controller input is also updated
with the informations about the largest force in the mesh, which is
used in the equation of the next time step, as explained in the nmesh
manual. The time\_step function set the new position of the nodes,
where the boundary nodes are moved in a way such that the value of
each of the boundary conditions on the node must be $<=$ to the
present values. Some of the nodes could move outside the object, and a
\om{\_enforce\_boundary\_conditions} function provides to move them back
on the closest boundary. At this point also the information on the
largest relative movement since last triangulation is updated.
The function do\_command associates the status of the controller to one
of the functions just described and the engine function checks for the
any request from the driver (to execute a callback function), any
stopping condition from the updated status of the and eventually calls
the do\_command function if none of the stopping conditions are met.     

\end{document}
